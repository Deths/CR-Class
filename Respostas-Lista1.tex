\documentclass[11pt,reqno]{amsart}
\usepackage[brazil]{babel}
\usepackage[utf8]{inputenc}
\usepackage{url}
\usepackage{lscape}
\usepackage{tabularx}
\usepackage{longtable}
\usepackage{setspace}
\usepackage[pdftex]{graphicx}
\usepackage{url}
\usepackage{mathtools}
\usepackage{pdfpages}
\graphicspath{ {C:/Users/user/Desktop/imagens/} }
\usepackage{verbatim}
\setcounter{MaxMatrixCols}{12}


\usepackage[a4paper,top=1cm,bottom=1.50cm,left=1.2cm,right=1.2cm]{geometry} %


\pagestyle{plain}

\def\aux{\mathop{\text{\rm aux}}\nolimits}

\usepackage{setspace}
\onehalfspacing



\title{Comunicação e redes}
\author{}

\date{}

\begin{document}



\begin{center}
Comunicação e redes - Lista 1\\ 
Name - Student's number \\ \ \\
\end{center}



\begin{enumerate}

\item

\begin{itemize}
	\item Um grafo G = (V,E) é uma estrutura composta por um conjunto V de vértices e um conjunto E de pares de vértices chamado de conjunto das arestas de G. Os grafos são uma representação visual de redes complexas, o que facilita o estudo das mesmas.
	\item As vantagens de usar a matriz adjacente é a rápida identificação das vizinhanças do vértice a ser analizado. As desvantagens se tratam de um grande espaço a ser usado para a confecção das matrizes, além de que, por ter simetria em relação a diagonal principal, muitos dados se tornam repetidos, usando memória desnecessariamente. \\ Já a lista adjacente tem a vantagem de ocupar pouco espaço quando o grafo possui poucas arestas. Já as desvantagens são a grande dificuladade em descobrir o vizinho de um dado vértice e vão ser longas quando se possui muitos vizinhos.
	\item Classificando as redes em quatro tipos, temos: \\ Redes sociais: Facebook, Twitter. \\ Redes de informações: Google, Bing. \\ Redes de transporte: CPTM, Uber. \\ Redes biológicas: Ser humano, ecossistema. 
\end{itemize}
\vspace{0.3cm}


\item 
\begin{itemize}
	\item A representação do \textit{Grafo de Petersen} por lista adjacência:\\\[A \xRightarrow[ ]{ } B \xRightarrow[ ]{ } E \xRightarrow[ ]{ } G \]\\\[B \xRightarrow[ ]{ } A \xRightarrow[ ]{ } C \xRightarrow[ ]{ } H \]\\\[C \xRightarrow[ ]{ } B \xRightarrow[ ]{ } D \xRightarrow[ ]{ } I \]\\\[D \xRightarrow[ ]{ } C \xRightarrow[ ]{ } E \xRightarrow[ ]{ } J \]\\\[E \xRightarrow[ ]{ } A \xRightarrow[ ]{ } D \xRightarrow[ ]{ } F \]\\\[F \xRightarrow[ ]{ } E \xRightarrow[ ]{ } H \xRightarrow[ ]{ } I \]\\ \[G \xRightarrow[ ]{ } A \xRightarrow[ ]{ } I \xRightarrow[ ]{ } J \]\\\[H \xRightarrow[ ]{ } B \xRightarrow[ ]{ } F \xRightarrow[ ]{ } J \]\\\[I \xRightarrow[ ]{ } C \xRightarrow[ ]{ } F \xRightarrow[ ]{ } G \]\\\[J \xRightarrow[ ]{ } D \xRightarrow[ ]{ } G \xRightarrow[ ]{ } H \]
	\item Representação do \textit{Grafo de Petersen} por matriz de adjacência:\\ \[ \begin{bmatrix*}[r] \ & A & B & C & D & E & F & G & H & I & J\\ A & 0 & 1 & 0 & 0 & 1 & 0 & 1 & 0 & 0 & 0 \\ B & 1 & 0 & 1 & 0 & 0 & 0 & 0 & 1 & 0 & 0\\ C & 0 & 1 & 0 & 1 & 0 & 0 & 0 & 0 & 1 & 0\\ D & 0 & 0 & 1 & 0 & 1 & 0 & 0 & 0 & 0 & 1\\ E & 1 & 0 & 0 & 1 & 0 & 1 & 0 & 0 & 0 & 0\\ F & 0 & 0 & 0 & 0 & 1 & 0 & 0 & 1 & 1 & 0\\ G & 1 & 0 & 0 & 0 & 0 & 0 & 0 & 0 & 1 & 1\\ H & 0 & 1 & 0 & 0 & 0 & 1 & 0 & 0 & 0 & 1\\ I & 0 & 0 & 1 & 0 & 0 & 1 & 1 & 0 & 0 & 0\\ J & 0 & 0 & 0 & 1 & 0 & 0 & 1 & 1 & 0 & 0 \end{bmatrix*} \]
\end{itemize}
\vspace{0.3cm}

\item Ordem: A ordem de um grafo G é a quantidade de vértices de G, denoatada por v0, isto é, vG = |V|.\\ Tamanho: O tamanho de G, denotado por eG, é a quantidade de arestas de G, isto é, eG = |E|\\ Grau de um vértice: O grau de um vértice M, denotado por dG (M), é a quantidade de arestas que incidem em M.\\ Vizinhança de um vértice: A vizinhança de um vértice se trata de todos os vértices conectados a o vértice a ser estudado por arestas.\\ Passeio: Um passeio em um grafo G = (V,E) é uma sequência alternada de vértices e arestas que começa e termina com vértices.\\ Trilha: Se não ocorre repetição de arestas em nosso passeio, ocorre o que denominamos de trilha.\\ Caminho: Caso não ocorra a repetição de vértices no passeio pelo grafo, ocorre o que denominamos de caminho.\\ Ciclo: Uma trilha fechada que também possui todos os vértices diferentes, menos os vértices iniciais e finais, é chamado de ciclo.\\ Trilha euleriana: Se trata de um trilha que passe por todas as arestas do grafo.\\ Caminho hamiltoniano: Em um grafo G é um caminho que passa por todos os vértices do grafo.\\ Ciclo hamiltoniano: É um ciclo que passa por todos os vértices de G.
\vspace{0.3cm}



\item 
\begin{itemize}
	\item Ordem: Temos 10 vértices distintos, logo a ordem do grafo é 10
	\item Tamanho: Temos 15 arestas distintas, logo o tamanho do grafo é 15
	\item Grau de todos os vértices: a = 3, b = 3, c = 3, d = 3, e = 3, f = 3, g = 3, h = 3, i = 3, j = 3.
	\item Conjunto dos vizinhos de cada vértice: a = (b,e,g), b = (a,c,h), c = (b,d,i), d = (c,e,j), e = (a,d,f), f = (e,h,i), g = (a,i,j), h = (b,f,j), i = (c,f,g), j = (d,g,h)
	\item De acordo com a análises feitas, não é possível construir um caminho euleriano, pois para isso é necessário que todo vértice interno do caminho tenha que ter grau par enquanto os externos tenham um grau ímpar. No nosso grafo, todos os vértices têm grau 3, impossibilitando a execução de um caminho euleriano.
	\item Não existe nenhum ciclo euleriano, pois, para iso, seria necessário que cada vértice do grafo conexo tivesse grau par.
	\item Um caminho Hamiltoniano possível:\\\[F \xRightarrow[ ]{ } E \xRightarrow[ ]{ } D \xRightarrow[ ]{ } C \xRightarrow[ ]{ } I \xRightarrow[ ]{ } G \xRightarrow[ ]{ } A \xRightarrow[ ]{ } B \xRightarrow[ ]{ } H \xRightarrow[ ]{ } J\]
\end{itemize}
\vspace{0.3cm}


\item \includegraphics{grafo}
\vspace{0.3cm}

\item 
\begin{itemize}
	\item O algoritmo funciona da seguinte forma:\\Primeiramente se elege o vértice gerador/pai,distancia = 0\\Em seguida cria-se uma filha vazia, onde se iniciará introduzindo o vértice pai\\Enquanto a fila não estiver vazia:\\ 1) Retiramos o elemento mais antigo da fila;\\ 2)Analisamos todos os vizinhos desse elemento;\\ 3) Se a distância entre esse vértice vizinho e o vértice pai for infinita(ou seja, ainda não definida, pois ainda não são convexos), agora agregamos um valor da distância do vértice que se decidiu analisar +1.\\ 4) Ao fim, adiciona-se esse elemento no fim da fila.
	\item Pois a fila faz analogia à árvore de busca e largura, em que se tem um único caminho mínimo entre 'S' e todos os vértices do grafo que é alcançável a partir de 'S'.
	\item A distância entre o vértice 'A' e todos os outros vértices é de no máximo 2.
\end{itemize}
\vspace{0.3cm}{}
{}
\item Para isto seria necessário realizar um algoritmo de busca em largura tendo o Ponto A como vértice pai/gerador esse algoritmo seria executado até o encontro do vértice destino 'B'. A menor quantidade de conexões entre vôos será dado pela distância de 'B' a 'A' nessa árvore gerada.




\end{enumerate}
\end{document}















