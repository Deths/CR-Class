\documentclass[11pt,reqno]{amsart}
\usepackage[brazil]{babel}
\usepackage[utf8]{inputenc}
\usepackage{url}
\usepackage{lscape}
\usepackage{tabularx}
\usepackage{longtable}
\usepackage{setspace}
\usepackage[pdftex]{graphicx}
\usepackage{url}
\usepackage{mathtools}
\usepackage{pdfpages}
\usepackage{verbatim}
\usepackage{setspace}
\usepackage{soul}
\usepackage{float}
\usepackage[a4paper,top=1cm,bottom=1.50cm,left=1.2cm,right=1.2cm]{geometry} %
\graphicspath{ {C:/Users/carva/Desktop/imagens/} }
\pagestyle{plain}
\def\aux{\mathop{\text{\rm aux}}\nolimits}
\onehalfspacing

\title {Lista 3 - Comunicações e Redes}
\author{}
\date{}

\begin{document}
	\begin{center}
		Comunicações e Redes - Lista 3\\
		Luís Kenzo Takayama - 11201721937 \\ \ \\
	\end{center}

	\begin{enumerate}

		\item

		\vspace{0.3cm}

		\begin{enumerate}
			\item Grafos aleatórios são muito usados para provar grafos por meio de métodos probabilisticos,
			 satisfazendo várias propriedades ou fornecendo definições rigorosas. Considere um grafo com 
			 N vértices, completo, agora, vamos deletando as arestas do nosso grafo completo, sendo p, a 
			 probabilidade daquela aresta não ser deletada.
			\item Não, como as arestas possuem a mesma probabilidade de serem geradas, elas se distribuem 
			muito bem pelo grafo a ser analisado, criando um efeito que denominamos, efeito de expansão.
			\item Temos que $|E(G(n,p))| \approx \frac{n(n-1)p}{2}$, logo, podemos fazer: \[n (log(n)) = 
			\frac{n(n-1)p}{2} \Rightarrow p = \frac{2 n(log(n))}{n(n-1)} \Rightarrow p = \frac{2(log(n))}{n-
			1}\] Logo o valor do meu p, nessa situação, seria $\frac{2(log(n))}{n-1}$
			\item Uma propriedade importante G(n,p) é a percolação, que ajuda na criação de um gráfico 
			robusto, isto porque independente do $n \gg 1$, considerando um grau médio K. é eliminado uma 
			fracção aleatória de 1- p' de vértices e deixa apenas uma fracção p' no nosso grafo.
			Porém existe um limiar crítico de percolação, P'\textit{c} = $\frac{1}{k}$, abaixo do qual 
			a rede se torna fragmentada, enquanto acima de P'\textit{c}, um componente ligado de grandes 
			dimensões de ordem n existe.
			\item Para que nosso grafo seja vazio, o nosso número p deve assumir valo zero, assim não
			há nenhuma probabilidade de uma aresta ser criada em nosso vértice. Já para ser um grafo 
			completo, o nosso p deve assumir valor 1, assim, todas as arestas possíveis tem total chance
			de serem geradas em nosso grafo.
		\end{enumerate}

		\vspace{0.3cm}

		\item
		\begin{minipage}{\linewidth}
		\centering
		\includegraphics{Resposta2L3}
		\end{minipage}
		
		\vspace{0.3cm}

		\item Uma propriedade que o grafo G(n,p) não satisfaz é a do "agrupamento", ou seja, a 
		probabilidade de que, em uma rede social, por exemplo, dois conhecidos seus se conheçam, é maior do 
		que pessoas randomicas se conheçam. Dado o número de vértices N, e sendo K o grau de cada vértice( 
		vamos assumir como sendo um número inteiro par) e $\beta$ sendo um parâmetro especial, satisfazendo
		$0 \leq \beta \leq 1$  $\wedge$  $N \gg K \gg ln(N) \gg 1 $, o modelo constrói um grafo com N 
		vértices e $\frac{NK}{2}$ arestas da seguinte forma: \\ 1) É criado uma rede de anéis onde cada 
		vértice possui grau 2K e está conectado a K vizinhos mais próximos em ambos os lados. \\ 2) Para
		cada aresta no gráfico, reconecte o vértice de destino com probabilidade $\beta$ de tal forma que 
		não haja uma conexão entre o vértice e ele mesmo, ou que um vértice não conecte duas vezes com um 
		outro vértice. 

		\vspace{0.3cm}

		\item 
		\begin{minipage}{\linewidth}
		\centering
		\includegraphics{Resposta4L3}
		\includegraphics{Resposta4L3-2}
		\end{minipage}

		\vspace{0.3cm}

		\item Temos que, todos os vértices do grafo gerado para esse exercício tem o coeficiente de 
		agrupamento igual, logo precisamos calcular apenas uma vez, sabendo que o vértice analisado se 
		conecta a $\frac{2n}{3}$, sendo que cada um desses vértice se liga a todos do outro grupo de 
		vértices, logo, temos $\frac{n^2}{9}$ arestas, logo: \[ C\textit{a} = \frac{\frac{n^2}{9}}{\frac{2n}
		{3}(\frac{2n-3}{3})}{2} \Rightarrow \frac{n^2}{9} X \frac{9X2}{4n^2-6n} \Rightarrow C\textit{a} = 
		\frac{n^2}{2n^2-3n} \] Temos que o coeficiente de agrupamento de um grafo obtido no passo 
		determinístico é: $C\textit{a} = \frac{3(k-2)}{4(k-1)}$. Se tendermos os dois ao 
		infínito, ou seja, para um número muito grande, teremos: \[\lim_{k\to\infty} \frac{3(k-2)}{4(k-1)} =
		 \frac{3}{4}\] \[\lim_{n\to\infty} \frac{n^2}{2n^2-3n} = \frac{1}{2}\] Logo o grafo do passo 
		 determinístico possui um coeficiente de agrupamento maior.

		\vspace{1cm}

		\item O modelo WS apresenta uma distribuição média característica de graus em seus vértices, porém, 
		isso não é comumente encontrado nas redes da natureza, em que temos uma distribuição distinta 
		(graus concentrados em alguns vértices, chamados de hub). Um modelo que grafo que trabalha com 
		essas desigualdades é o Rede de livre escala que segue uma lei de potência. \\
		Para a construção de um grafo de livre escala começaremos com um grafo de m0 vértices e a cada passo
		adicionamos um novo vértice que se conecta a m vértices do grafo obtido até o passo anterior. E 
		assumimos que a probabilidade do novo vértice se ligar com os vértices é proporcional ao grau 
		deste, favorecendo os Hubs.

		\vspace{0.3cm}


		\item O problema inicial que podemos citar é se o grafo inicial possui arestas, pois, para um novo 
		vértice se ligar ao nosso grafo antigo, ele possui uma probabilidade $\pi$ que está relacionado com 
		o grau do vértice alvo, isto é, do vértice do nosso grafo inicial, porém se todos os vértices de 
		nosso grafo possuirem grau 0, nunca haverá expansão. Outro problema é se o grau esperado do novo 
		vértice for 1, já que uma das regras do nosso grafo de livre escala é que esse novo vértice se ligue
		a m vértices do nosso grafo a ser analizado. 

		\vspace{0.3cm}

		\item  Quando m=1, isto é, adição de 1 aresta a cada passo e um vértice. O modo com a qual o 
		vértice é posto não importa, o que nos interessa são as ligações que ele pode fazer que interfere 
		na configuração do grafo. De forma aleatória, o vértice novo é conectado a outro vértice de acordo 
		com a seguinte fórmula: \[P(i=s)=\frac{di(vs)}{2t-1} \rightarrow Se 1 \leq s \leq t-1\]
		\[P(i=s) = \frac{1}{2t-1} \rightarrow Se s = t\] E assim se constrói um grafo de livre escala de 
		Bollobás, Riordan, Spenser e Tusnády.

		\vspace{0.3cm}

		\item O grafo $G^t_m$ é formado de $G^t_1$, identificando os m primeiros vértices para ser v1, em 
		seguida os m segundos vértices para ser v2 e assim por diante.

		\vspace{0.3cm}

		\item \begin{minipage}{\linewidth}
		\centering
		\includegraphics{Resposta10L3}
		\end{minipage}

	\end{enumerate}
\end{document}